% arara: xelatex: {shell: yes}
% %arara: biber
% %arara: xelatex: {shell: yes}
% %arara: xelatex: {shell: yes}

\documentclass[12pt]{article}

\usepackage{hyperref} % гиперссылки

\usepackage{tikz} % картинки в tikz
\usetikzlibrary{arrows.meta} % tikz-прибамбас для рисовки стрелочек подлиннее

\usepackage{microtype} % свешивание пунктуации

\usepackage{array} % для столбцов фиксированной ширины

\usepackage{indentfirst} % отступ в первом параграфе

\usepackage{sectsty} % для центрирования названий частей
\allsectionsfont{\centering}

\usepackage{amsmath} % куча стандартных математических плюшек
\usepackage{amssymb} % символы
\usepackage{amsthm} % теоремки

\usepackage{comment} % добавление длинных комментариев

\usepackage[top=2cm, left=1.2cm, right=1.2cm, bottom=2cm]{geometry} % размер текста на странице

\usepackage{lastpage} % чтобы узнать номер последней страницы

\usepackage{enumitem} % дополнительные плюшки для списков
%  например \begin{enumerate}[resume] позволяет продолжить нумерацию в новом списке

\usepackage{caption} % что-то делает с подписями рисунков :)

\usepackage{qcircuit} % для рисовки квантовых диаграмм
\usepackage{physics} % бракеты

\usepackage{answers} % разделение условий и ответов в упражнениях


\usepackage{fancyhdr} % весёлые колонтитулы
\pagestyle{fancy}
\lhead{Числа бывают\ldots{ }два}
\chead{}
\rhead{КЛШ-2023 (46 сезон)}
\lfoot{}
\cfoot{}
\rfoot{\thepage/\pageref{LastPage}}
\renewcommand{\headrulewidth}{0.4pt}
\renewcommand{\footrulewidth}{0.4pt}



\usepackage{todonotes} % для вставки в документ заметок о том, что осталось сделать
% \todo{Здесь надо коэффициенты исправить}
% \missingfigure{Здесь будет Последний день Помпеи}
% \listoftodos — печатает все поставленные \todo'шки



\usepackage{booktabs} % красивые таблицы
% заповеди из докупентации:
% 1. Не используйте вертикальные линни
% 2. Не используйте двойные линии
% 3. Единицы измерения - в шапку таблицы
% 4. Не сокращайте .1 вместо 0.1
% 5. Повторяющееся значение повторяйте, а не говорите "то же"



\usepackage{fontspec} % что-то про шрифты?
\usepackage{polyglossia} % русификация xelatex

\setmainlanguage{russian}
\setotherlanguages{english}

% download "Linux Libertine" fonts:
% http://www.linuxlibertine.org/index.php?id=91&L=1
\setmainfont{Linux Libertine O} % or Helvetica, Arial, Cambria
% why do we need \newfontfamily:
% http://tex.stackexchange.com/questions/91507/
\newfontfamily{\cyrillicfonttt}{Linux Libertine O}

\AddEnumerateCounter{\asbuk}{\russian@alph}{щ} % для списков с русскими буквами
\setlist[enumerate, 2]{label=\asbuk*),ref=\asbuk*}

%% эконометрические сокращения
\DeclareMathOperator{\Cov}{Cov}
\DeclareMathOperator{\Arg}{Arg}
\DeclareMathOperator{\Corr}{Corr}
\DeclareMathOperator{\Var}{Var}
\DeclareMathOperator{\E}{\mathbb{E}}
\newcommand \hVar{\widehat{\Var}}
\newcommand \hCorr{\widehat{\Corr}}
\newcommand \hCov{\widehat{\Cov}}
\newcommand \cN{\mathcal{N}}
\let\P\relax
\DeclareMathOperator{\P}{\mathbb{P}}

\usepackage{multicol}

\usepackage[bibencoding = auto,
backend = biber,
sorting = none,
style=alphabetic]{biblatex}

\addbibresource{forecast_everything.bib}



% делаем короче интервал в списках
\setlength{\itemsep}{0pt}
\setlength{\parskip}{0pt}
\setlength{\parsep}{0pt}




\Newassociation{sol}{solution}{solution_file}
% sol --- имя окружения внутри задач
% solution --- имя окружения внутри solution_file
% solution_file --- имя файла в который будет идти запись решений
% можно изменить далее по ходу
\Opensolutionfile{solution_file}[all_solutions]
% в квадратных скобках фактическое имя файла

% магия для автоматических гиперссылок задача-решение
\newlist{myenum}{enumerate}{3}
% \newcounter{problem}[chapter] % нумерация задач внутри глав
\newcounter{problem}[section]

\newenvironment{problem}%
{%
\refstepcounter{problem}%
%  hyperlink to solution
     \hypertarget{problem:{\thesection.\theproblem}}{} % нумерация внутри глав
     % \hypertarget{problem:{\theproblem}}{}
     \Writetofile{solution_file}{\protect\hypertarget{soln:\thesection.\theproblem}{}}
     %\Writetofile{solution_file}{\protect\hypertarget{soln:\theproblem}{}}
     \begin{myenum}[label=\bfseries\protect\hyperlink{soln:\thesection.\theproblem}{\thesection.\theproblem},ref=\thesection.\theproblem]
     % \begin{myenum}[label=\bfseries\protect\hyperlink{soln:\theproblem}{\theproblem},ref=\theproblem]
     \item%
    }%
    {%
    \end{myenum}}
% для гиперссылок обратно надо переопределять окружение
% это происходит непосредственно перед подключением файла с решениями



\theoremstyle{definition}
\newtheorem{definition}{Определение}



\begin{document}

\newcommand{\dayone}{
При умножении комплексных чисел их длины умножаются, а углы (аргументы) складываются.

\begin{enumerate}
  \item Храбро возведи в степень комплексные числа
  
  \begin{minipage}{\linewidth}
    \begin{multicols}{3}
      \begin{enumerate}    
      \item $(1 + i)^2$; 
      \item $i^2$; 
      \item $(1 - i)^5$;
    \end{enumerate}
  \end{multicols}
  \end{minipage}
  \item Упрости так, чтобы ежу было понятно
  
  \begin{minipage}{\linewidth}
    \begin{multicols}{3}
      \begin{enumerate}    
      \item $(2 + 3i)\cdot (1 - i)$; 
      \item $(2 + 5i) / (1 - i)$; 
      \item $1 + i + i^2 + i^3 + i^4 + i^5$;
    \end{enumerate}
  \end{multicols}
  \end{minipage}
  \item Реши уравнение в комплексных числах
  
  \begin{minipage}{\linewidth}
    \begin{multicols}{3}
      \begin{enumerate}    
      \item $z^2 = -1$; 
      \item $z^2 = i$; 
      \item $z^2 = 1 + i$; 
    \end{enumerate}
  \end{multicols}
  \end{minipage}
  \item Вечная черепаха всю жизнь движется по прямой. 
  В первый час своей жизни движется со скоростью 10 км/ч, затем каждый час её скорость падает на 20\%. 
  
  Какой путь черепаха пройдет за свою бесконечную жизнь?

  \item Вечный черепах стартует в начале координат. 
  Изначально ползёт вправо со скростью 10 км/ч, затем каждый час поворачивает на 90$^{\circ}$ влево 
  и снижает скорость на 20\%.
  
  В какую точку он стремится?
  
  \item Нарисуй и найди 
  
  \begin{minipage}{\linewidth}
    \begin{multicols}{2}
      \begin{enumerate}    
      \item $\arctan \frac{1}{2} + \arctan \frac{1}{3}$;
      \item $\tan(\arctan (n + 2) - \arctan (n+1))$;
      \item $\arctan \frac{1}{3} + \arctan \frac{1}{7} + \arctan \frac{1}{13} + \ldots +\arctan \frac{1}{n^2 + 3n + 3} + \ldots$;
    \end{enumerate}
  \end{multicols}
  \end{minipage}
  
\end{enumerate}
}

\newcommand{\daytwo}{
\begin{enumerate}
  \item Вспомни, чему равно $i^2$? А чему равно $(-i)^2$?
  \item Забыв о предрассудках, храбро реши уравнения в комплексных числах
  
  \begin{minipage}{\linewidth}
    \begin{multicols}{2}
      \begin{enumerate}    
    \item $x^2 + 2x + 5 = 0$;
      \item $x^2 - 6x + 10 = 0$;
      \item $(x - i - 2)(x + 2 + 3i) =0$;
      \item $x^2 - (2 + 3i)x -2 + 2i = 0$;
    \end{enumerate}
  \end{multicols}
  \end{minipage}
  \item Найди числа 
  
  \begin{minipage}{\linewidth}
    \begin{multicols}{2}
      \begin{enumerate}    
    \item $\tan(\arctan 32 - \arctan 17)$;
      \item $\tan(\arctan (1/5) + \arctan (1/7))$;
      \item $\cos(\arctan (1/5) - \arctan (1/7))$;
      \item $\sin(\arctan (1/5) + \arctan (1/7))$;
      \item $\tan(\arcsin (1/5) + \arcsin (1/7))$;
      \item $\tan(\arctan (n+2) - \arctan (n+1))$;
    \end{enumerate}
  \end{multicols}
  \end{minipage}

  \item Тебя ждут новые уравения в комплексных числах!
  
  \begin{minipage}{\linewidth}
    \begin{multicols}{3}
      \begin{enumerate}    
      \item $z^2 = -1$; 
      \item $z^2 = i$; 
      \item $z^2 = 1 + i$; 
      \item $z^3 = 1$;
      \item $z^6 = -1$;
      \item $z^4 = -i$;
    \end{enumerate}
  \end{multicols}
  \end{minipage}
  \item Улитка стартует в начале координат. 
  Изначально улитка ползёт вправо со скростью 1 км/ч, через час поворачивает на 90$^{\circ}$ против часовой стрелки 
  и снижает скорость в два раза. После поворота ползёт два часа, снова поворачивает на  90$^{\circ}$ против часовой стрелки и
  снова снижает скорость в два раза. После нового поворота ползёт три часа и так далее. 
  После очередного поворота ползёт на один час дольше со скоростью в два раза меньше предыдущей. 

  В какую точку стремится улитка?
  

\end{enumerate}
}


\newpage
\dayone
\vfill
\dayone

\newpage
\daytwo
\vfill
\daytwo
\newpage




\newpage

\section{Лог. КЛШ-2023}

\begin{enumerate}
  \item 
\end{enumerate}

В теховском файле \verb|\newpage| стоит, чтобы легко было скопировать секцию, для печати двух копий подряд на одном листе.
Это позволяет экономить бумагу и время при печати :)

\subsection{Плакат}





\Closesolutionfile{solution_file}

% для гиперссылок на условия
% http://tex.stackexchange.com/questions/45415
\renewenvironment{solution}[1]{%
         % add some glue
         \vskip .5cm plus 2cm minus 0.1cm%
         {\bfseries \hyperlink{problem:#1}{#1.}}%
}%
{%
}%



\section{Решения}
\input{all_solutions}


\section{Источники мудрости}

\todo[inline]{передалать потом в bib-файл}

\begin{enumerate}
\item \url{https://github.com/bdemeshev/probability_dna}
\item \url{https://github.com/bdemeshev/probability_pro}
\end{enumerate}

\printbibliography[heading=none]


\end{document}
